\documentclass[letter, twocolumn]{ieeeconf}
\usepackage{url}

\title{ {\small Practical English Skill II} \\ Makeup Reporting Assignment}
\author{48-166636, Kentaro Wada}

\begin{document}

\pagestyle{empty}
\maketitle
\thispagestyle{empty}
\sloppy

\section{Introduction}

I selected a TED talk, ``How we're teaching computers to understand pictures'' by Fei-Fei Li
\footnote{\url{http://www.ted.com/talks/fei_fei_li_how_we_re_teaching_computers_to_understand_pictures}}.
Because currently my research topic is the perception of the real world by robot
and the title of the talk interests me.

In the following sections,
I describe the abstract, my opinions and relationship with my research topic in detail.

\section{Abstract of the talk}

In the introduction part, the presenter talks about three-year-old children,
and the ability of image captioning with some example images on screen.
After that, she points out the poor ability of the current computers about that captioning task
compared to these children.
Her talk starts with posing a problem on current computers.
She also raises some situations where smarter computers, which would recognize the image,
could help people: navigate blind people, track change of forest, security camera, and so on.

Next, she goes down to more technical contents in the talk,
with explaining about why recognizing world from images is so difficult.
It is because image is just a two dimensional array of numbers,
and extracting meaning from these numbers is required to realize the meaning of the image.
She also talks about the traditional method to teach machines to recognize the world,
with her experiences in the university when she was a PhD. student.

After the description of these difficulties,
she gives the story about the decision to collecting huge number of images
to train machines to recognize the meaning of images.
That dataset, ``IMAGENET'', is one of the most largest dataset on both quality and quantity.

She moves to computer algorithms to train the computers using that training dataset,
with a short description about neural networks especially convolutional networks.
And the recent results by these algorithms on recognizing objects inside images.
But she also raises another task by computer that beyond recognizing object on images,
which is the description about the image as sentences.

She gives description about the algorithms to generate sentences from image,
that has a recurrent construction in the neural network model.
And the image captioning results by computer are given with example images on the screen.
She also some desirable progress in these captioning skills to break current limitation
of machine abilities.

In conclusion, she gives lots of situations where the image recognition technology can help human:
medicine, car driving, robot for rescue, and exploring in space.
And she closes the presentation with a message that machine can collaborate and help with human.

\section{My Opinion}

In this section, I describe my opinion, question, and feeling after listening that talk.

I was impressed by her skills of speech to describe technical stuffs to people
who are not so familiar with research topics.
The construction of presentation is similar to that of research paper: firstly raise problem
in current society and methodologies, give conventional methods with their limitations,
propose methods and its performance.
But she tries not to use technical terms, for example ``see'' instead of ``recognition'',
which is usually used in computer vision research papers.
She also use technique to shortly describe about technical stuff using a single image:
for example representing machine learning model as a box with gears.
This gives people to understand the word with the image or icon, and makes it easier.

A question I had while listening the talk was the description that
suitable algorithm to train the machine using large size of image dataset.
I can understand that convolutional network is one of the algorithm in deep learning methods
and suitable to process large number of images.
But I felt a jump in the explanation, and thought someone in the audience could not understand
about it. In my opinion, she can explain neural networks first,
and then describe convolutional network is one of the networks
which is known as a suitable model to process images.

\section{Relationship with my research}

In this section, I describe about the relationship between this TED talk and my research topic.

My research topic is 3D recognition of the real world, especially for robot tasks. Because
robot needs to understand what objects are located, what state it is
in order to conduct task in the real world.
Recognition of objects from image, which is the topic in this talk, is also one of the components
in the system I am constructing in the research.
In my system, robot recognize objects from image,
transform the result to 3D information, accumulate that information in time series,
plan to manipulate objects, and complete tasks.
So object recognition and understanding its state is the first component in the system,
and the topic in the talk is closely related to my research.

Processing big data which is described in the talk, is also related to my research topic.
I also use data-driven approach in the research to recognize the world three-dimensionally,
and a machine learning method, deep learning, described in the talk.
To generate the large scale dataset,
I use automation methods using robot to collect the sets of image and label.
In the talk, the presenter says the dataset is collected from web and annotated by human,
but I also use different approach using robot arms. That automated data collection system
is usually called self-supervised system, and a topic in robotic research field
to boost ability of machines, especially robots, to conduct tasks in the real world.

In the conclusion in the talk, she says about the collaboration of machine and human,
and machine can be more helpful to human.
My research purpose is also to help human by substituting some tasks of people by robot,
and I think this is also a point which is close to the topic in the talk.

\section{Conclusions}

In this report, I selected a TED talk, whose topic is ability of machine to recognize the world,
and described the abstract, opinion, and relationship with my research topics.

\end{document}
